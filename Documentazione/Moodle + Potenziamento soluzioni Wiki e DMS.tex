%\documentclass[12pt,a4paper,oneside]{report}
\documentclass[12pt,a4paper,oneside]{article} %se usiamo articole commentiamo \appendix
%\documentclass[a4paper, twoside]{article} %inizio documento, stampato su 2 lati, tipo articolo

\usepackage[italian]{babel} %imposta la lingua a italiano
\usepackage[utf8]{inputenc} %permette la compilazione delle lettere accentate

\usepackage{makeidx} %aggiunge il pakage degli indici
	\usepackage[breaklinks]{hyperref} %genera l'indice linkato
	\makeindex %genera l'indice

\usepackage{graphicx} %permette di importare i grafici in pdf e immagini
\usepackage{latexsym} %permette di inserire i caratteri speciali
\usepackage{multirow} %colonne a multiriga
\usepackage{verbatim} %commento a blocco usando \begin{comment} \end{comment}

\usepackage[toc,acronym]{glossaries}
	% --- ACRONIMI ---

\newacronym{cas}{CAS}{\gls{Central Authentication Service}}
\newacronym{dms}{DMS}{\gls{Document Management System}}
\newacronym{crud}{CRUD}{Create Read Update Delete}

% --- GLOSSARIO ---

\newglossaryentry{Central Authentication Service}
{
  name=Central Authentication Service,
  description={è un servizio single sign-on libero che permette alle applicazioni web la possibilità di rinviare tutte le autenticazioni a un server centrale o a più server di fiducia}
}
\newglossaryentry{Document Management System}
{
  name=Document Management System,
  description={è una categoria di sistemi software che serve a organizzare e facilitare la creazione collaborativa di documenti e di altri contenuti.\\
Tecnicamente il DMS è un'applicazione lato server che si occupa di eseguire operazioni massive sui documenti, catalogandoli ed indicizzandoli secondo determinati algoritmi}
}
	\makeglossaries

\pagestyle{plain} %stampa indice a piepagina

\title{Moddlepedia - Wikiddle - }

\author{Amodio\\Stefano\\727457
\and Pinzan\\Andrea\\732345
\and Racchetti\\Luca\\703311}

\date{\today}

\begin{document} %inizio del documento

%-------------------------------------------------------------------------

\begin{titlepage}
%\maketitle %genera la prima pagina
\begin{center}
		\Huge{Moddlepedia - Wikiddle}\\
	\vskip 20pt
	\begin{Large}
		\begin{tabular}{ c c c }
		  Amodio & Pinzan & Racchetti \\
		  Stefano & Andrea & Luca \\
		  727457 & 732345 & 703311 \\
		\end{tabular}
	\end{Large}\\
	\vskip 20pt
		\begin{normalsize}
			\today\\
		\end{normalsize}
	\vfill
		\includegraphics[scale=0.8]{Immagini/logo.jpg}
	\vfill
	\begin{normalsize}
		Versione 0.0.1	
	\end{normalsize}
\end{center}
\end{titlepage}

\thispagestyle{empty}
\newpage %nuova pagina

\tableofcontents
\appendix
\printindex % stampa l'indice
\newpage %nuova pagina
%-------------------------------------------------------------------------



%-------------------------------------------------------------------------

\section{Ambito e Scopo del documento e del progetto}

\gls{cas} è un acronimo la prima volta.\\
\gls{cas} è un acronimo la seconda volta.\\
\gls{dms}\\
\gls{Central Authentication Service} è un glossario la prima volta.\\
\gls{Central Authentication Service} è un glossario la seconda volta.

aaaaa \cite{esp1} bbbb \cite{esp1} cccc

\newpage %nuova pagina

%-------------------------------------------------------------------------
	
\section{Acronimi e Glossario}

\deftranslation{Glossary}{Glossario}
\deftranslation{Acronyms}{Acronimi}
\renewcommand*{\glossaryname}{Glossario}
\renewcommand*{\acronymname}{Acronimi}

\setglossarysection{subsection}

%\glsaddall %imposta la stampa di tutti i termini del gossario, anche se non utilizzati

%\addcontentsline{acronym}{subsection}{Acronimi}
%\printglossary[type=acronym]
%\printglossary[type=acronym,title=Acronimi]

%\addcontentsline{toc}{subsection}{Glossario}
%\printglossary[type=toc]
%\printglossary[type=toc,title=Glossario]

\printglossary[numberedsection,type=acronym]
\printglossary[numberedsection]
%\printglossaries

\newpage %nuova pagina

%-------------------------------------------------------------------------
\section{Documenti utilizzati}

\newpage %nuova pagina
%-------------------------------------------------------------------------

\section{Gap Analysis}

\newpage %nuova pagina
%-------------------------------------------------------------------------

\section{Modello degli oggetti organizzativi (Business Object Model)}

\subsection{Analisi di Dominio - Slide 13}

\newpage %nuova pagina	

%-------------------------------------------------------------------------

\section{Elenco dei requisiti funzionali e informativi}

% funzionale: cosa fa e non come, es. il sitema farà questo quando l'utente farà quell'latro
% non funzionale: 
% informativo: le informazioni gestite dal sitema es. input/output; è cosa è richiesto dal sistema

\subsection{Req.1 - Gestione documento}
\paragraph{Tipologia:}
	Funzionale
\paragraph{Descrizione:}
	L'utente deve poter effettuare il \gls{crud} su un documento.
\paragraph{Importanza:}
	N (Need)
\paragraph{Motivazione:}
	Le azioni di \gls{crud} su un documento sono necessarie per poter poterlo gestire.
\paragraph{Criterio di Validazione:}
	Il documento può essere creato?\\
	Il documento può essere modificato?\\
	Il documento può essere visualizzato?\\
	Il documento può essere cancellato?
\paragraph{Legame con altri requisiti:}
	La gestione di un documento è la base per requisiti successivi.

\subsection{Req.2 - Caricamento file utilizzabile nel documento}
\paragraph{Tipologia:}
	Funzionale
\paragraph{Descrizione:}
	L'utente deve poter caricare un file (immagine, pdf, audio, video, ...) utilizzabile nel corpo del documento.
\paragraph{Importanza:}
	N (Need)
\paragraph{Motivazione:}
	Rendere un file disponibile all'interno del documento a fine esplicativo.
\paragraph{Criterio di Validazione:}
	Il file può essere caricato?\\
	Il file è disponibile nel documento?
\paragraph{Legame con altri requisiti:}
	\subparagraph{Prerequisito} Req.1


\subsection{Req.3 - Gestione di gruppi di utenti}
\paragraph{Tipologia:}
	Funzionale
\paragraph{Descrizione:}
	Il docente può effettuare il \gls{crud} sui gruppi di utenti.
\paragraph{Importanza:}
	N (Need)
\paragraph{Motivazione:}
	Permette al docente di organizzare gruppi di lavoro.
\paragraph{Criterio di Validazione:}
	Il gruppo può essere creato?\\
	Si può assegnare un utente in un gruppo?\\
	Si può rimuovere un utente da un gruppo?\\
	Il gruppo può essere cancellato?
\paragraph{Legame con altri requisiti:}



\subsection{Req.4 - Gestione dei permessi sul documento}
\paragraph{Tipologia:}
	Funzionale
\paragraph{Descrizione:}
	Il docente può permettere ad un gruppo di utenti di visualizzare e/o modificare un documento.\\
	Può inoltre cambiare il gruppo assegnato al documento.
\paragraph{Importanza:}
	N (Need)
\paragraph{Motivazione:}
	Permette al docente di assegnare ad uno specifico gruppo un ambiente di lavoro.
\paragraph{Criterio di Validazione:}
	Al documento può essere associato un gruppo?\\
	Al documento può essere cambiato gruppo?\\
	Il gruppo può visualizzare il documento?\\
	Il gruppo può modificare il documento?\\
	Il documento non è accessibile agli esterni al gruppo assegnato?
\paragraph{Legame con altri requisiti:}
	\subparagraph{Prerequisito} Req.1
	\subparagraph{Prerequisito} Req.3


%\subsection{Req.5 - Gestione dei permessi sul file}
%	ereditarietà


\subsection{Req.5 - Gestione delle versioni di un documento}
\paragraph{Tipologia:}
	Informativo e Funzionale %chiedere al prof.
\paragraph{Descrizione:}
	Il sistema deve tenere traccia e visualizzare le modifiche effettuate sul documento con relativa data e autore.\\
	Gli utenti possono ritornare a una versione precedente del documento.
\paragraph{Importanza:}
	M (Must)
\paragraph{Motivazione:}
	Poter visualizzare l'evoluzione del documento.\\
	Poter ritornare a una versione precedente del documento.\\
	Poter verificare il contributo di ogni componente del gruppo sul documento.
\paragraph{Criterio di Validazione:}
	Quando un utente modifica il documento, la modifica viene tracciata?\\
	Viene visualizzata la cronostoria del documento?\\
	E' possibile ripristinare una versione precedente?
\paragraph{Legame con altri requisiti:}
	\subparagraph{Prerequisito} Req.1


\subsection{Req.6 - Gestione delle versioni di un file}
\paragraph{Tipologia:}
	Informativo e Funzionale %chiedere al prof.
\paragraph{Descrizione:}
	Il sistema deve tenere traccia e visualizzare gli upload di un file con relativa data e autore.\\
	Gli utenti possono ritornare a una versione precedente del file.
\paragraph{Importanza:}
	W (wish)
\paragraph{Motivazione:}
	Poter visualizzare l'evoluzione del file.\\
	Poter ritornare ad una versione precedente del file.\\
	Poter scaricare una qualunque versione del file.\\
	Poter verificare l'autore del caricamento del file.
\paragraph{Criterio di Validazione:}
	Quando un utente ricarica il file, viene tracciata la versione precedente?\\
	Viene visualizzata la cronostoria del file?\\
	E' possibile ripristinare una versione precedente?\\
	E' possibile scaricare una qualunque versione del file?
\paragraph{Legame con altri requisiti:}
	\subparagraph{Prerequisito} Req.1
	\subparagraph{Prerequisito} Req.2

\newpage %nuova pagina

%-------------------------------------------------------------------------	

\section{Casi d'uso e di interazione}

\subsection{CASO1}
\subsubsection{CASO1 - Obiettivo}
\subsubsection{CASO1 - Descrizione ordinata e dettagliata interazione fra Utenti o Utenti-Sistema}
\subsubsection{CASO1 - Scenari}
\paragraph{CASO1 - Happy Flow}
	\begin{itemize}
	\item acquisto iniziato e concluso con successo
	\end{itemize}
\paragraph{CASO1 - Percorsi alternativi}
	\begin{itemize}
	\item Disponiblità di un prodotto terminata
	\item Carta di credito non acettata
	\item Collegamento con servizi interbancari interrotto
	\end{itemize}
\subsubsection{CASO1 - Diagrammi}

\newpage %nuova pagina

%-------------------------------------------------------------------------

\section{Descrizione di un processo notevole di interazione}

\subsection{Modello/diagramma dinamico (processo e di collaborazione)}

\newpage %nuova pagina

%-------------------------------------------------------------------------
%-------------------------------------------------------------------------
%-------------------------------------------------------------------------
%-------------------------------------------------------------------------
%-------------------------------------------------------------------------
%-------------------------------------------------------------------------
%-------------------------------------------------------------------------
%-------------------------------------------------------------------------

\section{Diagramma}

\subsection{di Struttura}

\subsubsection{static structure diagram}
\subsubsection{entity diagram}
\subsubsection{class diagram}
\subsubsection{activity diagram / behavior}
\subsubsection{BPMN}

\newpage %nuova pagina

%-------------------------------------------------------------------------
	
\section{Appendici}

\subsection{diagrammi di sequenza}
\subsection{collaboration diagram}
\subsubsection{entity diagram + scambi di messaggio}

\newpage %nuova pagina

%-------------------------------------------------------------------------


\begin{thebibliography}{50}
 \bibitem{esp1} D. P. Woodruff, W. A. Royer, N. V. Smith (1986), \emph{Physical Review B} {\bf 34}, 2.
 \bibitem{esp2} K. Giesen, F. Hage, F. J. Himpsel, H. J. Riess, W. Steinmann (1986), \emph{Physical Review B} {\bf 33}, 8.
  \bibitem{esp3} M. Ortuno, P. M. Echenique (1986), \emph{Physical Review B} {\bf 34}, 8.
  \bibitem{A-B} J. A. Appelbaum, E. I. Blount (1973), \emph{Physical Review B} {\bf 8}, 483.
 \bibitem{Art:PG} J. B. ~Pendry, S. J. Gurman (1974), \emph{Surface Science } {\bf 49}, 87-105.   
 \bibitem{Art:Griffiths} David J.~Griffiths (2005), \emph{Introduzione alla Meccanica Quantistica}, Ambrosiana.
\bibitem{Acciarri} N. W. Ashcroft, N. D. Mermin (1976), \emph{Solid State Physics}, Holt, Rinehart, Winston.
	\bibitem {2bande} P. M. Echenique, J. M. Pitarke, E. V. Chulkov, V. M. Silkin (2008), \emph{Journal of electron spectroscopy and related phenomena} {\bf 126}, 163-175.
\bibitem{Art:Chulkov} E. V. ~Chulkov, V. M. Silkin, P. M. Echenique (1999), \emph{Surface Science} {\bf 437}, 330-352.
 \bibitem{Cohen:quantum1} C.~Cohen-Tannoudji, B.~Diu, F.Laloe (1973), \emph{Quantum Mechanics}, Vol.~1, Wiley-VCH.
\bibitem{bucatri}  J. H. Davies (1998), \emph{The Physiscs of Low Dimensional Semiconductors-An Introduction}, Cambridge University Press.
   \bibitem{Cohen:quantum2} C.~Cohen-Tannoudji, B.~Diu, F.Laloe (1973), \emph{Quantum Mechanics}, Vol.~2, Wiley-VCH.
 \bibitem{Art:EP} P. M. ~Echenique, J. B. Pendry (1977), \emph{Solid State Physics} {\bf 11}, 2065.
 \bibitem{Art:Smith} N. V. ~Smith (1985), \emph{Physical Review B} {\bf 32}, 6.
   \bibitem {} G.Butti (2005) \emph{PhD Thesis: Surface specific electronic structure on extended substrates}.

\end{thebibliography}


\end{document} %fine del documento
