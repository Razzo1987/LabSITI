% funzionale: cosa fa e non come, es. il sitema farà questo quando l'utente farà quell'latro
% non funzionale: 
% informativo: le informazioni gestite dal sitema es. input/output; è cosa è richiesto dal sistema

\subsection{Req.1 - Gestione documento}
\paragraph{Tipologia:}
	Funzionale
\paragraph{Descrizione:}
	L'utente deve poter effettuare il \gls{crud} su un documento.
\paragraph{Importanza:}
	N (Need)
\paragraph{Motivazione:}
	Le azioni di \gls{crud} su un documento sono necessarie per poter poterlo gestire.
\paragraph{Criterio di Validazione:}
	Il documento può essere creato?\\
	Il documento può essere modificato?\\
	Il documento può essere visualizzato?\\
	Il documento può essere cancellato?
\paragraph{Legame con altri requisiti:}
	La gestione di un documento è la base per requisiti successivi.

\subsection{Req.2 - Caricamento file utilizzabile nel documento}
\paragraph{Tipologia:}
	Funzionale
\paragraph{Descrizione:}
	L'utente deve poter caricare un file (immagine, pdf, audio, video, ...) utilizzabile nel corpo del documento.
\paragraph{Importanza:}
	N (Need)
\paragraph{Motivazione:}
	Rendere un file disponibile all'interno del documento a fine esplicativo.
\paragraph{Criterio di Validazione:}
	Il file può essere caricato?\\
	Il file è disponibile nel documento?
\paragraph{Legame con altri requisiti:}
	\subparagraph{Prerequisito} Req.1


\subsection{Req.3 - Gestione di gruppi di utenti}
\paragraph{Tipologia:}
	Funzionale
\paragraph{Descrizione:}
	Il docente può effettuare il \gls{crud} sui gruppi di utenti.
\paragraph{Importanza:}
	N (Need)
\paragraph{Motivazione:}
	Permette al docente di organizzare gruppi di lavoro.
\paragraph{Criterio di Validazione:}
	Il gruppo può essere creato?\\
	Si può assegnare un utente in un gruppo?\\
	Si può rimuovere un utente da un gruppo?\\
	Il gruppo può essere cancellato?
\paragraph{Legame con altri requisiti:}



\subsection{Req.4 - Gestione dei permessi sul documento}
\paragraph{Tipologia:}
	Funzionale
\paragraph{Descrizione:}
	Il docente può permettere ad un gruppo di utenti di visualizzare e/o modificare un documento.\\
	Può inoltre cambiare il gruppo assegnato al documento.
\paragraph{Importanza:}
	N (Need)
\paragraph{Motivazione:}
	Permette al docente di assegnare ad uno specifico gruppo un ambiente di lavoro.
\paragraph{Criterio di Validazione:}
	Al documento può essere associato un gruppo?\\
	Al documento può essere cambiato gruppo?\\
	Il gruppo può visualizzare il documento?\\
	Il gruppo può modificare il documento?\\
	Il documento non è accessibile agli esterni al gruppo assegnato?
\paragraph{Legame con altri requisiti:}
	\subparagraph{Prerequisito} Req.1
	\subparagraph{Prerequisito} Req.3


%\subsection{Req.5 - Gestione dei permessi sul file}
%	ereditarietà


\subsection{Req.5 - Gestione delle versioni di un documento}
\paragraph{Tipologia:}
	Informativo e Funzionale %chiedere al prof.
\paragraph{Descrizione:}
	Il sistema deve tenere traccia e visualizzare le modifiche effettuate sul documento con relativa data e autore.\\
	Gli utenti possono ritornare a una versione precedente del documento.
\paragraph{Importanza:}
	M (Must)
\paragraph{Motivazione:}
	Poter visualizzare l'evoluzione del documento.\\
	Poter ritornare a una versione precedente del documento.\\
	Poter verificare il contributo di ogni componente del gruppo sul documento.
\paragraph{Criterio di Validazione:}
	Quando un utente modifica il documento, la modifica viene tracciata?\\
	Viene visualizzata la cronostoria del documento?\\
	E' possibile ripristinare una versione precedente?
\paragraph{Legame con altri requisiti:}
	\subparagraph{Prerequisito} Req.1


\subsection{Req.6 - Gestione delle versioni di un file}
\paragraph{Tipologia:}
	Informativo e Funzionale %chiedere al prof.
\paragraph{Descrizione:}
	Il sistema deve tenere traccia e visualizzare gli upload di un file con relativa data e autore.\\
	Gli utenti possono ritornare a una versione precedente del file.
\paragraph{Importanza:}
	W (wish)
\paragraph{Motivazione:}
	Poter visualizzare l'evoluzione del file.\\
	Poter ritornare ad una versione precedente del file.\\
	Poter scaricare una qualunque versione del file.\\
	Poter verificare l'autore del caricamento del file.
\paragraph{Criterio di Validazione:}
	Quando un utente ricarica il file, viene tracciata la versione precedente?\\
	Viene visualizzata la cronostoria del file?\\
	E' possibile ripristinare una versione precedente?\\
	E' possibile scaricare una qualunque versione del file?
\paragraph{Legame con altri requisiti:}
	\subparagraph{Prerequisito} Req.1
	\subparagraph{Prerequisito} Req.2